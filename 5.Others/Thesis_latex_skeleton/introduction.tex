\chapter{Introduction}
In this thesis, search for the pair production of heavy Majorana neutrinos via the decay of new neutral gauge boson ($\Zp$) is presented. This search is mainly motivated by broken parity
symmetry of the Standard Model (SM) and smallness of neutrino masses. Event signature of the search consists of two same flavor leptons ($\Pe$ or $\Pgm$) and at least two jets. The data is collected
by the Compact Muon Solenoid (CMS) detector of the Large Hadron Collider (LHC) at CERN during 2016 to 2018, corresponding to an integral luminosity of 137.4 ${\unit{fb}}^{-1}$ from
proton-proton collisions at $\sqrt{s} = 13 \TeV$.

Models of heavy neutrinos based on the left-right symmetric extension of the SM (LRSM) are mainly inspired by unatural chiral structure of the weak interaction, and,
are suggested from about 40 years ago.  
Among left/right handed fields of fermions, charged gauge bosons (mediator) of the weak force (${\PW}^{\pm}$) interacts only with left-handed particles (right-handed anti-particles).
For this parity violation of the weak interaction, the SM provides no exact reason or source. The thoery will be more natural in the way of spontaneous symmetry breaking (SSB) of LRSM group
at certain high energy level to parity violated group, and it is key point of the LRSM.

In addition to this unnaturalness of chirality, there are additional puzzles on the SM related with masses of neutrinos.
After discovery of neutrino oscillations \cite{PhysRevLett.81.1158}, it is well known that at least two out of three mass eigenstates of neutrinos are non-zero.
In the SM, it is assumed that neutrinos are massless, so, neutrino oscillations are clear evidence for the physics beyond the SM. 
Furthermore, bounds on neutrino masses coming from direct measruements of tritium decays \cite{PhysRevLett.123.221802} is about 1.1 \ensuremath{\text{e\hspace{-.08em}V}}\xspace while
study on cosmology \cite{HUT1979144} gives about an order of smaller bound. Comparison between these small mass bound of neutrino masses and masses of other dirac particles of the SM gives
additional question as ``why masses of neutrinos are so tiny?''.

In the SM, fermions such as quarks or leptons get masses via Yukawa interactions between left-handed field, right-handed field, and Higgs field. With discovery of Higgs boson \cite{20121, 201230} and no
detection of right-handed neutrinos, the question if neutrinos get masses via same Yukawa interactions like other fermions or need another mechanism should be investigated.
The LRSM with complex Higgs sector, such as bi-doublet and two triplets for each chirality, can explain smallness of neutrino masses. Under good assumptions on vacuum expectation values
of Higgs fields, the model naturally adapts the see-saw mechanism \cite{PhysRevLett.44.912}. At same time, the LRSM group is naturally degraded into parity violated group,
in another word, to the group of the SM before SSB, and again to ${U(1)}_{em}$ group, the group of the SM after SSB.
As a result, the model is equivalent with the SM at the energy scale of current universe and beeing studied scale using particle colliders.

Besides limits from direct searches at particle colliders such as Tevatron and LHC Run1 on additional gauge bosons introduced by the LRSM (${\WR}^{\pm}$ and $\Zp$), most stringent limits are
coming from meson system with CP violation, mass difference between ${\PK}_{L}$ and ${\PK}_{S}$. For the worst case, the limit on mass of ${\WR}^{\pm}$ is given as to be greater than about 4 \TeVcc.
By using relation between mass of ${\WR}^{\pm}$ ($\mWR$) and mass of $\Zp$ ($\mZp$) which is $\mZp \simeq 1.7 \mWR$, limit on $\mZp$ becomes to be greater than about 6.8 \TeVcc. In the best case, 
limits are resolved to $(\mWR, \mZp) = (2.5 \TeVcc, 4.25 \TeVcc)$ from $(4 \TeVcc, 6.8 \TeVcc)$ \cite{PhysRevD.82.055022}. Either cases are accessible at the LHC Run2 proton-proton collisions with $\sqrt{s} = 13 \TeVcc$

Although addtional gauge bosons of the LRSM can be searched by low energy experiments such as B meson decays (B anomalies) and neutrionless double beta decays ($0\nu\beta\beta$) 
via new physcis contributions coming from $\WR$ and flavor changing heavy Higgs, direct search at the LHC has unique characteristic which is possibility of mass reconstruction of additional gauge
bosons and heavy neutrinos. Especially, direct measurement of masses of heavy neutrinos for multi flavor channels provides values of elements of Majorana mass matrix that are very important
physical parameters to understand results of low energy experiments.

The discovery of heavy neutrinos at the LHC will provide wider understanding on symmetries of the universe and the detailed characteristics of neutrino sector which is still the world of many unknowns.
Furthermore, hints on matter dominance of the universe in the sense of matter and anti-matter asymmetry would be also provided by additional CP violation source coming from parity symmetry breaking.
Cosmology based on three neutrino scheme and nuclear physics based on left-handed charged current weak interaction only would also get huge impact for further understanding on underlying physics.

This thesis mainly descibes the search for pair production of such heavy neutrinos via s-channel production of the new neutral gauge boson ($\Zp$) at the LHC based on the LRSM. The thesis is
structured as follows: The brief summary on history related with the SM and neutrino physics with detailed theories on them are discussed in chapter 2 including motivations
of the search. In chapter 3, brief review about the LHC is discussed including accelerator chain and simple principle of synchrotrons. The summary on the CMS detector is covered at chapter 4 also with
particle flow (PF) algorithm which is used for reconstruction of incoming particles from proton-proton collisions and triggering systems. In chapter 5, Monte Carlo simulations for signal processes and
backgroun processes are decribed. 

Chapter 6 is for data samples used for the analysis. Definitions of physical objects and event selection criteria are discussed in chapter 7.
Chapter 8 presents signal efficiencies under event selections of this analysis.
Correction applied to physical object are summarized in chapter 9. Background esimations are described in
chapter 10, and validation of those estimations and systematic uncertainties are discussed in chapter 11. The comparison between the data and the expected background in the signal region,
and result of the search are presented at chapter 12. Finally, in chapter 13, the thesis finishes with the conclusions.
