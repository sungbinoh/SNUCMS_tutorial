% 서울대학교 전기공학부 (전기컴퓨터공학부) 석사ㅡ 박사 학위논문
% LaTeX 양식 샘플
\RequirePackage{fix-cm} % documentclass 이전에 넣는다.
% oneside : 단면 인쇄용
% twoside : 양면 인쇄용
% ko : 국문 논문 작성
% master : 석사
% phd : 박사
% openright : 챕터가 홀수쪽에서 시작
\documentclass[oneside,phd,openright]{snuthesis}

%%%%%%%%%%%%%%%%%%%%%%%%%%%%%%%%%%%%%%%%%%%%%%%%%%%%%%%%%%%%%%%%%%%%%%%%%%%%%%
%%
%% Author: zeta709 (zeta709@gmail.com)
%% Releasedate: 2017/07/20
%%
%% 목차 양식을 변경하는 코드
%% * subfigure (subfig) package 사용 여부에 따라
%%   tocloft의 옵션을 다르게 지정해야 한다.
%% * Chapter 번호가 두 자리 수를 넘어가는 경우 다음과 같이
%%   필요한 만큼 "9"를 추가하면 된다.
%%   \settowidth{\mytmplen}{\bfseries\cftchappresnum9\cftchapaftersnum}
%%   아니면 \cftchapnumwidth를 직접 적당히 고치면 된다.
%%%%%%%%%%%%%%%%%%%%%%%%%%%%%%%%%%%%%%%%%%%%%%%%%%%%%%%%%%%%%%%%%%%%%%%%%%%%%%
%\usepackage[titles,subfigure]{tocloft} % when you use subfigure package
\usepackage[titles]{tocloft} % when you don't use subfigure package
\makeatletter
\if@snu@ko
	\renewcommand\cftchappresnum{제~}
	\renewcommand\cftchapaftersnum{~장}
	\renewcommand\cftfigpresnum{그림~}
	\renewcommand\cfttabpresnum{표~}
\else
	\renewcommand\cftchappresnum{Chapter~}
	\renewcommand\cftfigpresnum{Figure~}
	\renewcommand\cfttabpresnum{Table~}
\fi
\makeatother
\newlength{\mytmplen}
\settowidth{\mytmplen}{\bfseries\cftchappresnum\cftchapaftersnum}
\addtolength{\cftchapnumwidth}{\mytmplen}
\settowidth{\mytmplen}{\bfseries\cftfigpresnum\cftfigaftersnum}
\addtolength{\cftfignumwidth}{\mytmplen}
\settowidth{\mytmplen}{\bfseries\cfttabpresnum\cfttabaftersnum}
\addtolength{\cfttabnumwidth}{\mytmplen}
\makeatletter
\g@addto@macro\appendix{%
	\addtocontents{toc}{%
		\settowidth{\mytmplen}{\bfseries\protect\cftchappresnum\protect\cftchapaftersnum}%
		\addtolength{\cftchapnumwidth}{-\mytmplen}%
		\protect\renewcommand{\protect\cftchappresnum}{\appendixname~}%
		\protect\renewcommand{\protect\cftchapaftersnum}{}%
		\settowidth{\mytmplen}{\bfseries\protect\cftchappresnum\protect\cftchapaftersnum}%
		\addtolength{\cftchapnumwidth}{\mytmplen}%
	}%
}
\makeatother
 % SNU toc style

%%%%%%%%%%%%%%%%%%%%%%%%%%%%%%%%%%%%%%%%
%% 다른 패키지 로드
%% http://faq.ktug.or.kr/faq/pdflatex%B0%FAlatex%B5%BF%BD%C3%BB%E7%BF%EB
%% 필요에 따라 직접 수정 필요
\ifpdf
	\input glyphtounicode\pdfgentounicode=1 %type 1 font사용시
	%\usepackage[pdftex,unicode]{hyperref} % delete me
	\usepackage[pdftex]{graphicx}
	%\usepackage[pdftex,svgnames]{xcolor}
        \usepackage{subfig}
        \usepackage{xspace}
        \usepackage{amsmath,amssymb,cancel} %% -- To write formula
        \usepackage{multirow} %% -- To use multirow on tables
        \usepackage{pennames} %% -- From CMS note util
        \usepackage{ifthen} %% -- For ptdr-definitions
        \usepackage{ptdr-definitions} %% From CMS note util
\else
	%\usepackage[dvipdfmx,unicode]{hyperref} % delete me
	\usepackage[dvipdfmx]{graphicx}
	%\usepackage[dvipdfmx,svgnames]{xcolor}
\fi
%%%%%%%%%%%%%%%%%%%%%%%%%%%%%%%%%%%%%%%%

%\usepackage[
%    backend=biber,
%    style=authoryear-icomp,
%    sortlocale=de_DE,
%    natbib=true,
%    url=false, 
%    doi=true,
%    eprint=false
%]{biblatex}
\usepackage[backend=biber]{biblatex}
%\bibliography{Main.bib}
\addbibresource{Main.bib}
\usepackage[]{hyperref}
\hypersetup{
%    colorlinks=true,
}
\usepackage{lipsum} % lorem ipsum

%% -- You can define any character here for convenience
\newcommand{\N}{\ensuremath{\mathrm{N}}\xspace}
\newcommand{\mN}{\ensuremath{m_\mathrm{N}}\xspace}
\newcommand{\Zp}{\ensuremath{{Z}^{\prime}}\xspace}
\newcommand{\mZp}{\ensuremath{{m}_\mathrm{{Z}^{\prime}}}\xspace}
\newcommand{\mWR}{\ensuremath{{m}_\mathrm{{W}_{\mathrm{R}}^{\prime}}}\xspace}
\newcommand{\WR}{\ensuremath{{\mathrm{W}}_{\mathrm{R}}}\xspace}
\newcommand{\Ne}{\ensuremath{\mathrm{N}_{\Pe}}\xspace}
\newcommand{\Nm}{\ensuremath{\mathrm{N}_{\mu}}\xspace}
\newcommand{\Nt}{\ensuremath{\mathrm{N}_{\tau}}\xspace}
\newcommand{\WRpm}{\ensuremath{W_{\text{R}}^{\pm}}\xspace}
\newcommand{\SU}{\ensuremath{\mathrm{SU}}\xspace}
\newcommand{\U}{\ensuremath{\mathrm{U}}\xspace}
\newcommand{\mZpReco}{\ensuremath{{M}_\mathrm{{Z}^{\prime}}^{\text{Reco}}}\xspace}

%% \title : 22pt로 나오는 큰 제목
%% \title* : 16pt로 나오는 작은 제목
\title{Searching for heavy neutrinos using the LHC proton-proton collision data at $\sqrt{s} = 13~\TeV$ collected by the CMS detector.}
\title*{CMS 검출기로 수집된 거대 강입자 충돌기의 질량중심 에너지 $13~\TeV$ 양성자-양성자 충돌 데이터를 사용한 무거운 중성미자 탐사}

\academicko{이학}
\schoolen{COLLEGE OF NATURAL SCIENCE}
\departmenten{DEPARTMENT OF PHYSICS AND ASTRONOMY}
\departmentko{물리천문학부}

%% 저자 이름 Author's(Your) name
%\author{홍길동}
%\author*{홍~길~동} % Insert space for Hangul name.
\author{오성빈}
\author*{오~성~빈} % Same as \author.

%% 학번 Student number
\studentnumber{2013-20374}

%% 지도교수님 성함 Advisor's name
%% (?) Use Korean name for Korean professor.
%\advisor{홍길동}
%\advisor*{홍~길~동} % Insert space for Hangul name.
\advisor{양운기}
\advisor*{양~운~기}

%% 학위 수여일 Graduation date
%% 표지에 적히는 날짜.
%% 학위 수여일이 아니라 논문 발간년도를 적어야 할 수도 있음.
%\graddate{2010~년~2~월}
\graddate{June 2021}

%% 논문 제출일 Submission date
%% (?) Use Korean date format.
\submissiondate{2021~년~6~월}

%% 논문 인준일 Approval date
%% (?) Use Korean date format.
\approvaldate{2021~년~6~월}

%% Note: 인준지의 교수님 성함은
%% 컴퓨터로 출력하지 않고, 교수님께서
%% 자필로 쓰시기도 합니다.
%% Committee members' names
\committeemembers%
{최선호}%
{양운기}%
{김선기}%
{정성훈}%
{최수용}
%% Length of underline
%\setlength{\committeenameunderlinelength}{7cm}

\begin{document}
\pagenumbering{Roman}
\makefrontcover
\makefrontcover
\makeapproval

\cleardoublepage
\pagenumbering{roman}
% 초록 Abstract

\keyword{SNU, High Energy Physics, thesis}

\begin{abstract}
Among several unveiled key questions about the universe, neutrinos' masses and its mechanism are clear evidence beyond the Standard Model (SM).
In addition, unique chiral structure of the weak interaction of the SM is unnatural 
in the sense of parity violation without certain source of it.
The left-right symmetric extension of the SM is traditional way to explain the parity violation with specific Higgs sector, for example, a bi-doublet and two triplets,
which takes the see-saw mechanism into the model in natural way that can explain the smallness of neutrino masses. The model's ${SU(2)}_{L} \times {SU(2)}_{R} \times {U(1)}_{B - L}$ group
is reduced to the group indentical to the SM, ${SU(2)}_{L} \times {U(1)}_{Y}$, by spontaneous symmetry breaking (SSB) of Higgs sector, and, once again to ${U(1)}_{em}$ also by SSB.

In this thesis, the search for pair production of heavy neutrinos via the decay of new neutral gauge boson ($\Zp$) in the proton-proton collisions at $\sqrt{s} = 13 \TeV$ in same flavor di-lepton ($\Pe$ or $\Pgm$) plus at least two jets channel is presented.
The data set corresponds to the integrated luminosity of $137.4 {\unit{fb}}^{-1}$ collected by the Compact Muon Solenoid (CMS) detector at the Large Hadron Collider (LHC).

\end{abstract}


\tableofcontents
\listoffigures
\listoftables

\cleardoublepage
\pagenumbering{arabic}

\chapter{Introduction}
In this thesis, search for the pair production of heavy Majorana neutrinos via the decay of new neutral gauge boson ($\Zp$) is presented. This search is mainly motivated by broken parity
symmetry of the Standard Model (SM) and smallness of neutrino masses. Event signature of the search consists of two same flavor leptons ($\Pe$ or $\Pgm$) and at least two jets. The data is collected
by the Compact Muon Solenoid (CMS) detector of the Large Hadron Collider (LHC) at CERN during 2016 to 2018, corresponding to an integral luminosity of 137.4 ${\unit{fb}}^{-1}$ from
proton-proton collisions at $\sqrt{s} = 13 \TeV$.

Models of heavy neutrinos based on the left-right symmetric extension of the SM (LRSM) are mainly inspired by unatural chiral structure of the weak interaction, and,
are suggested from about 40 years ago.  
Among left/right handed fields of fermions, charged gauge bosons (mediator) of the weak force (${\PW}^{\pm}$) interacts only with left-handed particles (right-handed anti-particles).
For this parity violation of the weak interaction, the SM provides no exact reason or source. The thoery will be more natural in the way of spontaneous symmetry breaking (SSB) of LRSM group
at certain high energy level to parity violated group, and it is key point of the LRSM.

In addition to this unnaturalness of chirality, there are additional puzzles on the SM related with masses of neutrinos.
After discovery of neutrino oscillations \cite{PhysRevLett.81.1158}, it is well known that at least two out of three mass eigenstates of neutrinos are non-zero.
In the SM, it is assumed that neutrinos are massless, so, neutrino oscillations are clear evidence for the physics beyond the SM. 
Furthermore, bounds on neutrino masses coming from direct measruements of tritium decays \cite{PhysRevLett.123.221802} is about 1.1 \ensuremath{\text{e\hspace{-.08em}V}}\xspace while
study on cosmology \cite{HUT1979144} gives about an order of smaller bound. Comparison between these small mass bound of neutrino masses and masses of other dirac particles of the SM gives
additional question as ``why masses of neutrinos are so tiny?''.

In the SM, fermions such as quarks or leptons get masses via Yukawa interactions between left-handed field, right-handed field, and Higgs field. With discovery of Higgs boson \cite{20121, 201230} and no
detection of right-handed neutrinos, the question if neutrinos get masses via same Yukawa interactions like other fermions or need another mechanism should be investigated.
The LRSM with complex Higgs sector, such as bi-doublet and two triplets for each chirality, can explain smallness of neutrino masses. Under good assumptions on vacuum expectation values
of Higgs fields, the model naturally adapts the see-saw mechanism \cite{PhysRevLett.44.912}. At same time, the LRSM group is naturally degraded into parity violated group,
in another word, to the group of the SM before SSB, and again to ${U(1)}_{em}$ group, the group of the SM after SSB.
As a result, the model is equivalent with the SM at the energy scale of current universe and beeing studied scale using particle colliders.

Besides limits from direct searches at particle colliders such as Tevatron and LHC Run1 on additional gauge bosons introduced by the LRSM (${\WR}^{\pm}$ and $\Zp$), most stringent limits are
coming from meson system with CP violation, mass difference between ${\PK}_{L}$ and ${\PK}_{S}$. For the worst case, the limit on mass of ${\WR}^{\pm}$ is given as to be greater than about 4 \TeVcc.
By using relation between mass of ${\WR}^{\pm}$ ($\mWR$) and mass of $\Zp$ ($\mZp$) which is $\mZp \simeq 1.7 \mWR$, limit on $\mZp$ becomes to be greater than about 6.8 \TeVcc. In the best case, 
limits are resolved to $(\mWR, \mZp) = (2.5 \TeVcc, 4.25 \TeVcc)$ from $(4 \TeVcc, 6.8 \TeVcc)$ \cite{PhysRevD.82.055022}. Either cases are accessible at the LHC Run2 proton-proton collisions with $\sqrt{s} = 13 \TeVcc$

Although addtional gauge bosons of the LRSM can be searched by low energy experiments such as B meson decays (B anomalies) and neutrionless double beta decays ($0\nu\beta\beta$) 
via new physcis contributions coming from $\WR$ and flavor changing heavy Higgs, direct search at the LHC has unique characteristic which is possibility of mass reconstruction of additional gauge
bosons and heavy neutrinos. Especially, direct measurement of masses of heavy neutrinos for multi flavor channels provides values of elements of Majorana mass matrix that are very important
physical parameters to understand results of low energy experiments.

The discovery of heavy neutrinos at the LHC will provide wider understanding on symmetries of the universe and the detailed characteristics of neutrino sector which is still the world of many unknowns.
Furthermore, hints on matter dominance of the universe in the sense of matter and anti-matter asymmetry would be also provided by additional CP violation source coming from parity symmetry breaking.
Cosmology based on three neutrino scheme and nuclear physics based on left-handed charged current weak interaction only would also get huge impact for further understanding on underlying physics.

This thesis mainly descibes the search for pair production of such heavy neutrinos via s-channel production of the new neutral gauge boson ($\Zp$) at the LHC based on the LRSM. The thesis is
structured as follows: The brief summary on history related with the SM and neutrino physics with detailed theories on them are discussed in chapter 2 including motivations
of the search. In chapter 3, brief review about the LHC is discussed including accelerator chain and simple principle of synchrotrons. The summary on the CMS detector is covered at chapter 4 also with
particle flow (PF) algorithm which is used for reconstruction of incoming particles from proton-proton collisions and triggering systems. In chapter 5, Monte Carlo simulations for signal processes and
backgroun processes are decribed. 

Chapter 6 is for data samples used for the analysis. Definitions of physical objects and event selection criteria are discussed in chapter 7.
Chapter 8 presents signal efficiencies under event selections of this analysis.
Correction applied to physical object are summarized in chapter 9. Background esimations are described in
chapter 10, and validation of those estimations and systematic uncertainties are discussed in chapter 11. The comparison between the data and the expected background in the signal region,
and result of the search are presented at chapter 12. Finally, in chapter 13, the thesis finishes with the conclusions.


\printbibliography

\appendix
\chapter{First Appendix Section}
\label{sec:first_app_sec}

Okay.


\keywordalt{서울대학교, 고에너지물리학, 졸업논문}
\begin{abstractalt}
  입자물리학은 20세기를 거쳐 눈부신 발전을 이뤄냈지만 여전히 이 우주에는 밝혀지지 않은 수수께끼들이 있다. 그 중에서도 중성미자의 질량과 그것을 생성하는 기작은 표준모형을 넘어서는 이론이 필요하다고
  주장하는 대표적인 현상 중 하나다. 이에 덧붙여서, 약한상호작용이 가지는 특이한 카이랄성은 반전성 대칭 붕괴에 대해 확실한 근거를 제공하지 못한다는 점에서 자연스럽지 않다.
  표준모형의 좌우대칭 확장은 이러한 문제를 해결하는 전통적인 방법 중 하나다. 여기에 다소 복잡한 힉스보존 구조를 사용하면, 예를들어 하나의 겹이중항과 두개의 삼중항, 중성미자 질량이
  다른 페르미 입자들보다 왜 크게 작은지 설명할 수 있는 시소(see-saw)기작이 자연스럽게 적용된다. 이 확장된 모형의 군은 ${SU(2)}_{L} \times {SU(2)}_{R} \times {U(1)}_{B - L}$ 로 표현되며,
  자발적 대칭성 붕괴를 통해 낮은 에너지 수준에서는 표준모형과 동일한 ${SU(2)}_{L} \times {U(1)}_{Y}$ 군으로 근사된다. 여기서 한번더 자발적 대칭성 붕괴에 의해 ${U(1)}_{em}$ 군으로 표현된다.

  이 논문에서는 질량중심 에너지 13 \TeV 으로 발생시킨 양성자-양성자 충돌에서 위 모형에서 등장하는 새로운 중성 게이지 보존 (\Zp)이 생성되고, 이것이 두개의 무거운 중성미자로 붕괴하는 반응을
  탐사한 결과를 발표한다. 이 과정에서 두개의 하전 경입자와 네개의 젯들이 형성되는데, 하전 경입자들의 경우 두개의 전자 혹은 두개의 뮤온이 생성되는 과정만 고려되었다. 분석에 사용한 데이터는
  거대 강입자 가속기에 있는 CMS 검출기로 수집하였으며, 총 $137.4 {\unit{fb}}^{-1}$ 의 총 적분 광도량에 해당한다.
  
\end{abstractalt}

\acknowledgement
Thanks.

\end{document}

