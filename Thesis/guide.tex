\documentclass[a4paper]{report}
\usepackage{kotex}
\usepackage{ifpdf}
\usepackage{indentfirst}
\ifpdf
	\input glyphtounicode\pdfgentounicode=1 %type 1 font사용시
	\usepackage[pdftex,unicode]{hyperref} % delete me
	\usepackage[pdftex]{graphicx}
	\usepackage[pdftex,svgnames]{xcolor}
\else
	\usepackage[dvipdfmx,unicode]{hyperref} % delete me
	\usepackage[dvipdfmx]{graphicx}
	\usepackage[dvipdfmx,svgnames]{xcolor}
\fi

\title{전기공학부 논문 \LaTeX 템플릿 매뉴얼}

\author{어느 대학원생}

\begin{document}
\maketitle

\tableofcontents

\chapter{사용법}
\TeX, \LaTeX, kotex의 기본적인 사용법은 인터넷 등을 참고하자.
여기서는 snueethesis.cls의 사용법만 간단히 소개할 것이다.
이 템플릿은 pdflatex에 최적화되어 있다.
아마도 latex과 dvipdfmx의 조합도 사용할 수 있을 것이다.
그런데, pdflatex을 쓰면 pdf의 종이 크기가 알아서
될 것이고, dvipdfmx를 쓰면 별도의 옵션을 주고 사용해야 할 것이므로
편하게 pdflatex을 쓰는 것을 권장한다.
개인적으로 latex, dvips, ps2pdf의 조합은 별로 권장하지 않는다.
혹시 xelatex을 사용하겠다면 고칠 것이 조금 있을 것 같다.
\section{컴파일}
간단하게 pdflatex을 하자. 경우에 따라 두 번 이상 컴파일해야 한다.
\section{패키지 의존성}
가급적이면 의존 패키지를 줄이는 방향으로 갔다.

이 양식은 kotex 패키지를 사용한다.
영문 논문에서도 한글을 써야하기 때문에 그냥 넣었다.
사실 kotex 없이 한글을 사용하는 것도 불가능한 것은
아닌 것으로 알지만, 여기서는 편의를 위해 kotex 패키지를 사용하고 있다.

그리고 indentfirst, ifpdf 패키지를 사용한다.
문제가 된다면 패키지 로드 부분을 주석처리하면 된다.

몇몇 패키지는 cls에서 로드하지 않고 tex 파일에서
직접 로드하게 만들었으니, sample.tex도 잘 참고하자.

hyperref 패키지는 인쇄시에는 도움도
안 되고, 오히려 번거로운 사태가 발생할 수도 있으니
학위 논문을 위해서라면 그냥 안 쓰는 편이
나을 수도 있다.
\section{기타}
제공되는 sample.tex을 보면
사용법을 대충 이해할 수 있을 것 같다.

페이지 번호 양식을
클래스 파일에서 처리하는 것보다 tex에서
처리하는 것이 여러모로 나은 점이
있어서 tex에서 처리하도록 하였다.
아무튼 pagenumbering 명령 이전에
cleardoublepage 명령을 넣어야
홀짝수가 제대로 맞는다.
단면을 사용하는 경우에도 이렇게 하자.

저자 명령인 \textbackslash author와 \textbackslash author*의 경우,
전자는 스페이스를 안 넣은 이름, 후자는 스페이스를 넣은 이름이다.
하지만 영문 이름과 한글 이름을 섞어 쓰고 싶은 경우도 있다.
예를 들면 표지에는 영문 이름, 인준지에는 한글 이름을 넣고
싶을 수도 있다.
이 경우 다음과 같이 중간에 \textbackslash author를 재정의하면 간단하게
해결된다!
\textbackslash author\{Name\} \textbackslash author*\{Name\}
\textbackslash makefrontcover
\textbackslash author\{이름\} \textbackslash authro*\{이 름\}
\textbackslash makeapproval

\chapter{석박사 논문 양식}

전기공학부 졸업논문 양식\cite{under,grad}\을 최대한 따랐다.
또한, thesisonline의 양식 역시 참고했다.
두 양식이 엇갈리는 경우가 많은데,
이 경우 대체로 전기공학부 양식을 우선하였다.
양식에 명시되지 않은 것은 저자 마음대로 하였다.
사실, 양식에 모든 부분이 세밀하고 엄밀하게
정의되어 있는 것은 아니다.
상황에 따라 클래스 파일을 직접 고치는 일이
필요할 수도 있을 것이다.

최대한 가벼운 클래스 파일을 만드는 것을
목표로 하였고, 의존 패키지를 최소화하였다.

버그를 제보해주시면 매우 감사하겠다.

\section{표지}
영문 논문의 인준지 양식은 \cite{grad}의 pp.~12--13,
국문 논문의 인준지 양식은 \cite{grad}의 pp.~33,~35을 참고하였다.

상단 여백 3cm, 좌우 여백 2cm, 하단 여백 3cm으로 맞췄다.
(버전 0.99에서 하단 여백을 3.5cm으로 고쳤다.)

\section{원문제공 동의서}
이건 안 만들었다. 중도에 제출하는 4부 중 단 한 부에만
동의서가 들어가는데 이것 때문에 제본을 따로 해야
되는지 여부를 잘 모르겠다.

\section{인준지}
영문 논문의 인준지 양식은 \cite{grad}의 pp.~14--15,
국문 논문의 인준지 양식은 \cite{grad}의 pp.~36--37을 참고하였다.

인준지부터는 좌우 여백이 늘어난다.
상단 여백 3cm, 좌우 여백 2.5cm, 하단 여백 2cm으로 맞췄다.
(버전 0.99에서 하단 여백을 2.5cm으로 고쳤다.)
표지의 좌우 여백과 인준지의 좌우 여백이 다르기 때문에
제목의 줄바꿈 위치가 달라질 수 있다.
같은 위치에서 제목 줄바꿈을 하려면 title을 정의할 때
적절한 위치에 \textbackslash\textbackslash를 넣자.

이상한 것이 국문 양식에서는 한글 제목 22pt, 영문 제목 16pt가 되고,
영문 양식에서는 영문 제목과 한글 제목이 모두 16pt라는 것이다.
일단 양식을 충실이 따랐다.

그리고 국문 양식에서는 일부 여백이 명시되지 않았고,
영문 양식에는 0.5cm로 되어 있는데, 이것은 0.5cm으로 통일하였다.
그러나 국문 양식은 대충 봐도 간격이 0.5cm이 아니다.
정확하게 정의되지 않은 것 같은데, 필요에 따라
알아서 수정해야 될지도 모르겠다.

인준지에는 날짜가 두 개 들어가는데 
\textbackslash submissiondate와 \textbackslash approvaldate이다.
영문 양식이든 한글 양식이든 이 날짜는 한글을 사용하는 것으로
되어 있는 것으로 보인다.

\section{본문}
상단 여백 3cm, 좌우 여백 2.5cm, 하단 여백 3.5cm으로 맞췄다.

\section{감사의 글}
규정에 따르면 학교에 제출하는 논문에는 감사의 글을 넣지 않는다고
적혀 있는데, 감사의 글을 넣고 제본한 논문을 본 것 같기도 하다.

\section{이름}
이름은 조금 귀찮은 것이 멋을 부리기 위해 스페이스를
중간에 넣어두는 경우가 있다. 이것은 *이 붙은 명령과
안 붙은 명령으로 구분하고 있다.

\section{들여 쓰기}
영문 양식에서는 첫 단락을 들여 쓰기하지 않고
국문 양식에서는 첫 단락을 들여 쓰기하도록
만들었다.
인위적으로 \textbackslash noindet를 주지 않아도
그렇게 될 것이다.
그리고 옵션으로 indentfirst와 noindentfirst를 제공하기도 한다.

\section{챕터 스타일과 섹션 스타일}
국문 논문 양식에서 장 제목은 16pt로 해야 한다.
그러나 ``목차'', ``표 목차'', ``그림 목차'' 등은
22pt로 해야한다. 그런데 잘 생각해보면 ``목차''
같은 것도 특별한 장 제목 아닌가?
실제로 \LaTeX에서는 이 상황에서 \textbackslash chater*을
사용한다. 아무튼 
이 양식을 맞추기 위해 \textbackslash chapter*을
재정의하였는데, 혹시라도 본문 중간에 숫자 없는 장
제목을 사용하면 양식의 통일성을 해치게 되는 문제가 있다.

그리고 국문 논문의 영문 초록에서는 ``Abstract''를 22pt로
적고 있는데, 영문 논문의 국문 초록에서는 ``초록''을
16pt로 적고 있다. 이런 상황을 보면 국문 장 제목은
16pt, 영문 장 제목은 22pt인데, ``목차'' 류만 특이하게
국문에서도 22pt로 정해진 듯하다.

그리고 국문 양식에서는 ``감사의 글''에 16pt를 사용하고 있다.
영문 양식에서는 규정이 없느데 22pt를 써야 될 것 같다.

그리고 장 제목과 절 제목에서 줄 간격을 0으로 하라고
되어 있는데, 줄 간격 0은 오타인 것 같아서
1.0으로 해두었다. 혹시 1.0이 너무 좁아
보인다면 수정하면 된다.

장 제목의 위, 아래 간격은 나온대로 맞추었다.
정확히는 줄 간격 때문에 지정된 크기보다
간격이 조금 더 벌어진다. 예를 들어
장 제목 아래로 2cm를 두어야 하는데,
본문 자체에서도 줄 간격이 있으므로,
자로 재보면 2cm보다 약간 더 커진다는
것이다. 이것은 doc로 제공되는 템플릿에서도
마찬가지인데, 자로 재보면 2cm보다 약간 크다.
그런 관계로 억지로 2cm을 정확히 맞추지는 않았다.
그러나 영문 서식의 경우 ``Chapter'' 위로
종이 여백을 포함하여 6cm을 두게 되어 있는데,
이것은 워드 템플릿에서 자로 재보면
딱 6cm이 된다. 이런 것들은 6cm을 정확히 맞추었다.

절 제목의 위, 아래 간격에도 깨알같은 규정이
있으나(\cite{grad} pp.~22--23) 너무 깨알같은 관계로 이건 조정하지 않았다.
이것 역시 줄간격을 제외하고 0.6cm등을 맞추라는 의도
같은데, 줄간격을 넣고 하다보면 어긋날 수 밖에 없다.
그리고 \TeX에서는 stretchability와 shrinkability를
사용해서 간격을 조절하고 있는데, 정확하게 맞추는
것이 말처럼 쉽지도 않고,
필자의 능력에 벗어나는 일이기도 하다.
(기존의 템플릿에서도 이것까지 고치진
않은 것은 마찬가지였다.)

\section{글자 크기와 줄 간격}
학부에서 제공하는 한컴 한글 양식과 MS 워드 양식도
글자 크기나 줄 간격이 완벽하게 동일하지는 않다.
줄간격은 폰트에 따라, 영문이냐 한글이냐에 따라
다르게 나오는 경우도 있다. 줄간격이 이상해 보인다면
필요에 따라 직접 고치면 될 것이다.
아무튼 한 페이지에 25행이 나오게 맞추긴
했는데 완벽할지는 모르겠다\footnote{사실
25행을 맞추면 170\%가 아니다.}.
아무튼 cls 파일에서 linespread 값을
바꾸면 원하는 본문 줄 간격을
얻을 수 있다.

\section{표와 그림}
LaTeX에서는 원래 표와 그림 이름의 delimiter로 세미콜론이 들어간다.
하지만 \cite{grad}의 양식에는 안 들어간다.
일단 양식대로 세미콜론을 제거하였는데,
구분자가 없는 것이 약간 어색해 보인다.

그 외에 caption의 정렬이나 마침표 사용에 대해서도
규정이 있는데 클래스 파일에서 처리할 문제는 아니다.

\section{목차}
표 목차와 그림 목차 중 어느 것이 먼저 와야 하는
문제가 있는데 영문 양식과 국문 양식에서 순서가 다르다.
아무튼 이것을 바꾸려면 listoffigures와 listoftables
순서만 바꾸면 되므로 어렵지 않다.

목차 양식을 맞추기 위해 tocloft 패키지를 사용했는데,
이 패키지는
subfigure 패키지 사용 여부에 따라 옵션을 다르게 주어야 한다.
이런 저런 문제로 cls에 넣지 않고 따로 적었다.

\begin{thebibliography}{00}
\addcontentsline{toc}{chapter}{\bibname}

\bibitem{under} 서울대학교 공과대학 전기공학부,
학사 학위논문 작성지침,
\url{http://eei.snu.ac.kr/bbs/view.php?id=pds&no=156}

\bibitem{grad} 서울대학교 공과대학 전기공학부,
석사,박사 학위논문 작성지침,
\url{http://eei.snu.ac.kr/bbs/view.php?id=pds&no=170}

\bibitem{aaa} 학위논문 제본 및 인쇄 요령

\bibitem{bbb}
\url{http://thesisonline.snu.ac.kr}
\end{thebibliography}

\end{document}

